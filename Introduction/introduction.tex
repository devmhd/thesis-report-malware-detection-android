\chapter{Introduction}
\label{chap:intro}


\section{Introduction}\label{sec:int}
Smart phones and tablets are the most popular and widely used personal electronics devices today. According to a statistics, Android dominated the market of these devices with an 82.8\% share in 2015 Q2\cite{mrr:soms}. Due to Android’s vast user base, open nature and relatively less restrictive application distribution system, it has always been an attractive platform for malwares. According to a recent report published jointly by Kaspersky Labs and INTERPOL\cite{mrr:kasinter}, 20\% of devices that uses their software were attacked at least once by malware. So, it is an extreme need to develop an efficient malware detection system in Android.


\section{Motivation and Problem Statement}
Malware is a program which disrupts computer operation, gather sensitive information or gain access to private systems without users’ consent. With the ever increasing use of mobile devices, mobile malware pose a significant threat because these devices store contacts, bank account numbers, credit / debit numbers, private photos, messages and a lot of other sensitive information that can be leaked. The Kaspersky Lab study ‘Financial Cyberthreats in 2014’ reports that the number of financial malware attacks against Android users grew by 3.25 times in 2014. During 2014, Kaspersky Lab’s Android products blocked a total of 2,317,194 financial attacks against 775,887 users around the world. The lion’s share of these (2,217,979 attacks against 750,327 users) used Trojan-SMS malware, and the rest (99,215 attacks against 59,200 users) used Trojan-Banker malware \cite{mrr:finatt}. Given the tremendous growth of Android malware, there is a pressing need for effective malware detection methods.
%%---------------- Motivations & Objective --------------------
\section{Objectives}

Existing detection methods can be classified into two major categories: static (code analysis) and dynamic (runtime/behavioral analysis). The sneakiest malware are almost impossible to detect using static analysis, because they often obfuscate the malicious code using random keys. Some malware download the malicious code at runtime and remove it after execution \cite{mrr:damce}. In these cases, a code analysis for known malware signature cannot detect the malware.\\
These exists a few static and dynamic malware detection methods in the literature. Chandramohan et al.~\cite{mrr:dmmw} has given a high-level overview of various detection methods. Zhou et al.~\cite{mrr:damce} collected, classified and published a large collection of 1260 Android malware. We used malware samples from their collection to evaluate our detection method. Isohara et al.~\cite{mrr:kbaamd} demonstrated a system-call logging based method.\\
The objective of this paper is to demonstrate two detection methods for finding malware in Android. The first one is based on network traffic analysis. The method we described will be effective against malware that communicates with known malicious remote servers. The second one is based on system call analysis. 


\section{Contributions}
Our network traffic analysis is based on logging the URLs of all remote locations that are contacted by applications for a specific period of time. Given, we have a database of known malicious domains; the applications that contact any of those malicious domains can be flagged as malware. On the other hand, in our system call analysis, if we can log all system calls made by an application, we can try to use it on known malwares to find patterns in sequence of system calls. These signatures can be used to detect new applications infected by known malwares. For logging system calls, we use strace, a standard unix tool.\\

We described our detection method in a detailed step-by-step manner, mentioning all the necessary tools and techniques used. Also, we briefly explained the purpose behind each step. This paper can be used as a technical guideline by researchers, who are trying to develop network traffic-based or system call based malware detection applications.

\section{Organization of the Dissertation}

