\chapter{Introduction}
\label{chap:intro}

Next generation networks are gradually converging towards the all-IP
networks which can enable true global mobility and Internet
connectivity to mobile devices. 

\section{Introduction}\label{sec:int}
Internet Protocol (IP) is the underlying communication protocol that
allows an end host to get connected to other hosts over the public
Internet. 
%
Therefore, to facilitate continuous Internet connectivity for mobile
nodes, Internet Engineering Task Force (IETF) proposed Mobile
IPv6~\cite{msh:mipv6:rfc6275}, an IP-based mobility protocol.

This aggregated mobility management can significantly reduce signaling requirement and power
consumption. 

\section{Motivation and Problem Statement}
In a mobile computing environment, a number of \emph{network
parameters} (such as, network size, mobility rate, traffic rate)
influence the signaling costs related to mobility management. With
the rapid growth and popularity of mobile and wireless networks,

Finally, mobility protocols can be vulnerable to security threats.

%%---------------- Motivations & Objective --------------------
\section{Objectives}

The \emph{objectives} of this research are as follows:
\begin{itemize}
\item The first objective of this research
is to perform a comprehensive cost and scalability evaluation of the
%
\item The second objective of this research is the quantitative
evaluation of survivability of the mobility infrastructure and the
associated components. 
%
\item The fourth objective of this research is to protect mobility
protocols from security threats. 
%
\item Finally, mobility protocols require a realistic mobility model
that can mimic the movement pattern of nodes in motion. 
\end{itemize}

%%----------------------- Contributions ------------------
%%----------------------- Contributions ------------------

\section{Contributions}
The \emph{contributions} of the dissertation are summarized as
follows:

\begin{itemize}
\item Perform entity-wise cost evaluation of host and network mobility protocols.
%
\item Perform quantitative scalability analysis of host and network mobility protocols.
%
\item Perform multi-class queuing analysis and propose  a
dynamic scheduling algorithm to protect  crucial control messages
(of mobility management) against losses.
\end{itemize}

\section{Organization of the Dissertation}
The rest of the dissertation is organized as follows.
Chapter~\ref{chap:bg} presents a review of host and network
mobility protocols. 
%
