Today mobile computing has become a necessity and we are witnessing
explosive growth in the number of mobile devices accessing the
Internet. To facilitate continuous Internet connectivity for nodes
and networks in motion, mobility protocols are required and they
exchange various signaling messages with the mobility infrastructure
for protocol operation. Proliferation in mobile computing  has
raised several research issues for the mobility protocols.
%
First, it is essential to perform cost and scalability analysis of
mobility protocols to find out their resource requirement to cope
with future expansion. Secondly, mobility protocols have
survivability issues and are vulnerable to security threats, since
wireless communication media can be easily accessible to intruders.
%
The third challenge  in mobile computing is the protection of
signaling messages against losses due to high bandwidth requirement
of multimedia in mobile environments.
%
However, there is lack of existing works that focus on the
quantitative analysis of cost, scalability, survivability and
security of mobility protocols.

In this dissertation, we have performed comprehensive evaluation of
mobility protocols.
%
We have presented tools and methodologies required for the cost,
scalability, survivability and security analysis of mobility
protocols.
%
We have proposed a dynamic scheduling algorithm to protect mobility
signaling message against losses due to increased multimedia traffic
in mobile environments and have also proposed a mobile network
architecture that aims at maximizing bandwidth utilization.
%
The analysis  presented in this work can help network engineers
compare different mobility protocols quantitatively, thereby choose
one that is reliable, secure, survivable and scalable.
