Android, the fastest growing mobile operating system began its journey with the release of the Android beta in November 2007. At the moment this mobile OS boasts of a staggering 1 Billion users, who without a careful monitoring of their in-device-security, are susceptible to malicious applications hacking into their personal data. There are plenty repositories that are full of free and cracked versions of premium applications and repackaged applications, majority of which are infected with malicious behaviors. People, being always eager to use free contents, are often deliberately putting themselves in danger of data-hacking and other harmful services through downloading these malwares.\\

Our goal, therefore, was to investigate the nature and identity of a malicious application and devise a detection procedure based on them. We concentrated on a behavioral analysis of malwares, focusing on their identifiable traits and data-flows in the Android system. Our first approach was network-based, we captured the outgoing data packets and analyzed their source and destination, thereby filtering any malicious domain servers or repositories. Our second step was to identify certain system calls and their frequency in a malicious application to establish a threshold that measures the acceptability of a random application. In both cases, we ran our experiments on 1260 malwares, acquired from Android Malware Genome Project, a malware database created by Y. Zhou et al.\cite{mrr:damce} and 227 non-malware benign applications. Both procedures are described in this dissertation along with their corresponding results. Our hope is the analysis given here will provide security professionals with more definitive and quantitative approaches in their investigations of mobile malwares on Android system.
