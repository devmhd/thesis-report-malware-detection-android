\chapter{Conclusion}
\label{chap:conclusion}

\section{Summary}\label{sec:conc}
In this thesis, we proposed a detection method based on behavioral analysis of malwares in android. We did this presentation in several chapters, a summary of which is given below.\\

In the first Chapter, we defined our motivation behind the research, as well as the problem statement. We also mentioned our objectives and stated our contributions.\\
   
In the second chapter, we analyzed all related works, their summary, strength of their work and limitations. We gave a brief description to each in the first section and in the second, we presented a comparison between our work and all those aforementioned. \\

In the third chapter, we demonstrated first step to design a detection method, the network traffic analysis. In the first section and Second, we respectively summarize and elaborate our method which is as follows. Firstly, we record all the outgoing HTTP packets and creates dump file which contain information of which port is accessing which URI. We periodically executed netstat command throughout the duration of packet dumping and saved output. It gives information of which port number is being used by which application. Now, we aggregate these two maps to create a time-sequenced log of application and the URLs each application tried to contact (App-URL table). We search the URLs in the App-URL table for known malicious domains. If an application tries to connect to a rogue domain (URL), we flag it as a malware. We can also enrich our blacklist by adding other domains contacted by a flagged application. In the final section, we showed the findings.\\

In the fourth chapter, we discussed our second approach for malware detection which uses system call traces of applications to predict malicious activity. The process was as follows- we use a set of applications consisting both known and unknown malwares as the training dataset. The system call trace of a single application is a list of system calls the application used during execution. After collecting system call traces, we aggregate this traces to create two binary relation matrices $M_{mal}$ and $M_{nmal}$. Then, we calculate the Goodness rating of $j^{th}$ syscall. From it, we define the Goodness Rating of that application and classify the app as malware or non-malware. We also define and calculate some metrics in this chapter.\\

In the fifth chapter, based on the theory described in the previous chapter, we went into action. We showed a continuous process, algorithm to code (reiterated in Appendix).\\

In the sixth chapter, we showed the result obtained from our given model. We used tables and graphs to present our findings in a detailed manner.\\ 

\section{Future Works}
\label{sec:fuwo}

This section is a speculation on how our thesis titled ``Behavioral Malware Detection Approaches for Android'' is a precursor to many research possibilities in the future.

\subsection{In Network-based Analysis}
Our work in network-based detection method is one of the few definitive models among the available ones. The findings obtained show as follows- 
\begin{itemize}
\item Our network analysis results shows successful capture and identification of malwares connecting to black listed domains. In the future attempts of finding such event, the illustrated method can be implemented quite easily.
\item The network analysis method described here is for batch processing (can monitor multiple application at the same time), the future works depending on a network analysis of large data, our approach is scalable.
\item In any research related to finding specific domain address for a specific app, malware or not, our method can be applicable as a generic approach.
\item We used several Linux commands, the necessity/nature of which can be understood clearly form our example, and such knowledge is useful while working with Linux system in the future works.  
\end{itemize} 

\subsection{In System-Call Analysis}
Next we discuss possibilities our system-call based malware detection approach can generate. 

\begin{itemize}
\item Ours is one of the most basic and straightforward approaches available on system-call based detection methods, so it can be used as a part of a large detection method.
\item Our System call based approach outlines a threshold for acceptability of an android app (``Good'' or ``Bad'' as mentioned in the previous chapters), which represents a \emph{Novel} approach, and can be used for experimentation related to malwares.
\item In standardized testing for android applications, our system call approach can play a vital role.
\item Our work is focused on system call frequency, which is extracted by Linux commands, so this approach described here is very useful for data collection. We are going to upload all our scripts in a public repository for better access.     
\end{itemize}    

\subsection{In General Malware Detection}
Finally, our thesis as a whole can be used for following applications-

\begin{itemize}
\item Repositories (Google Play and others) can use our work as standard testing before allowing upload.
\item Malware Detection Application for android can be quite easily produced based on our model, even some of our codes can be directly used and/or manipulated into a detection software.
\item The field of malware detection is vast and ever-changing, so we touched only a portion of available malwares and malwares to come in the near future. Our work as a whole can be used as a filter to lessen the domain set.
\item Our data can be used for training other malware/non-malware apps.    
\end{itemize}    
