\chapter{Conclusion}
\label{chap:concl}

\section{Summary}\label{sec:conc}
In this thesis, we proposed a detection method based on behavioral analysis of malwares in android. We did this presentation in several chapters, a summary of which is given below.\\

In the first Chapter, we defined our motivation behind the research, as well as the problem statement. We also mentioned our objectives and stated our contributions.\\
   
In the second chapter, we analyzed all related works, their summary, strength of their work and limitations. We gave a brief description to each in the first section and in the second, we presented a comparison between our work and all those aforementioned. \\

In the third chapter, we demonstrated first step to design a detection method, the network traffic analysis. In the first section and Second, we respectively summarize and elaborate our method which is as follows. Firstly, we record all the outgoing HTTP packets and creates dump file which contain information of which port is accessing which URI. We periodically executed netstat command throughout the duration of packet dumping and saved output. It gives information of which port number is being used by which application. Now, we aggregate these two maps to create a time-sequenced log of application and the URLs each application tried to contact (App-URL table). We search the URLs in the App-URL table for known malicious domains. If an application tries to connect to a rogue domain (URL), we flag it as a malware. We can also enrich our blacklist by adding other domains contacted by a flagged application. In the final section, we showed the findings.\\

In the fourth chapter, we discussed our second approach for malware detection which uses system call traces of applications to predict malicious activity. The process was as follows- we use a set of applications consisting both known and unknown malwares as the training dataset. The system call trace of a single application is a list of system calls the application used during execution. After collecting system call traces, we aggregate this traces to create two binary relation matrices $M_{mal}$ and $M_{nmal}$. Then, we calculate the Goodness rating of $j^{th}$ syscall. From it, we define the Goodness Rating of that application and classify the app as malware or non-malware. We also define and calculate some metrics in this chapter.\\

In the fifth chapter, based on the theory described in the previous chapter, we went into action. We showed a continuous process, algorithm to code (reiterated in Appendix).\\

In the sixth chapter, we showed the result obtained from our given model. We used tables and graphs to present our findings in a detailed manner.\\ 

\section{Future Works}
\label{sec:fuwo}

This section is a speculation on how our thesis titled ``Behavioral Malware Detection Approaches for Android'' is a precursor to many research possibilities in the future.

\begin{itemize}
\item \emph{\textbf{In Network-based Analysis:}} Our work in network-based detection method is one of the few definitive models among the available.  

\end{itemize} 

