\chapter{Literature Survey}
\label{chap:lit}

In this chapter we have summarized all the related works. At first, we briefly discussed a survey on mobile malwares and their generic detection methods\cite{mrr:mmedd}. Then another survey with detection methods divided into two categoriesm static and dynamic \cite{mrr:dmmw}, followed by a time-line analysis and characterization of 1260 malware dataset \cite{mrr:damce} are summarized. We then proceed on to more focused approaches like network analysis \cite{mrr:damnl}, aggregated system call events \cite{mrr:dmamabasce} and finally system call patterns in repackaged applications \cite{mrr:iamratscs}. Our problem domain identification and detection approaches are indebted to these mentioned researches. Each research are presented in following subsection, and in the next subsection, a comparison-based literature is provided, posing all differences between all these related works and our approach.     

\section{Briefly described related works}

\subsection{Mobile Malware Evolution, Detection and Defense \cite{mrr:mmedd}} 
\begin{itemize}
\item The Research briefly summarizes the history of mobile malware, specifics of mobile security when compared with computer security, various attack vector and attack models, various detection techniques for specific mobile devices, the defense mechanisms to control mobile malware. 
\item \emph{Strength of the work:}\\ 
1. Works as a baseline for related research.\\
2. Clearly distinguishes among various malicious applications (e.g grey-wares etc.)\\
3. Defines various detection approaches.\\
4. Brings up newer models of malware posting like mal-advertising etc. to focus
\item \emph{Limitations:} It doesn't specify a definitive approach, rather generic.
\end{itemize}

\subsection{Detection of Mobile Malware in the Wild \cite{mrr:dmmw}}
\begin{itemize}
\item In this paper a survey of techniques that are used to detect mobile malware in the wild is presented and the limitations of current techniques are discussed. 
\item \emph{Strength of the work:} \\
1. Successfully classifies some malwares according to their behavior. \\
2. Clearly outlines static and dynamic analysis. \\
3. Suggests cloud based detection, resulting in off-device analysis for battery-life efficiency. 
\item \emph{Limitations:} Suggests that permission analysis can be used for pre-screening, where in the android system, without giving permission, an application cannot be installed to check its behavior.\\\\ 
\end{itemize}

\subsection{Dissecting Android Malware: Characterization and Evolution \cite{mrr:damce}}
\begin{itemize}
\item There are three goals and contributions of this paper. First, this paper presents the first large collection of 1260 Android malware samples (that we used in our research, see Acknowledgement) in 49 different malware families, which covers the majority of existing Android malware, ranging from August 2010 to October 2011. Second, based on the collected malware samples, it performs a timeline analysis of their discovery and characterize them based on their detailed behavior breakdown, including the installation, activation, and payloads. Third, it performs an evolution-based study of representative Android malware, which shows that they are rapidly evolving and existing anti-malware solutions are seriously lagging behind.
\item \emph{Strength of the work:}\\ 
1. Successful creation of data tables for malwares clearly stating their repository (official market/free repository); classification based on installation,activation, malicious payloads, permission issues; evolution of specific malwares through time (since they're discovered)\\
2. Evaluation of performance for 4 known antivirus on Android phone. \\
\item \emph{Limitations:} Permission comparison among different apps (malwares and non-malwares) which is inadequate to form definitive outcome, as we have learnt in our research.  
\end{itemize}


\subsection{Detecting Android Malware on Network Level \cite{mrr:damnl}}
\begin{itemize}
\item The paper’s analysis of packet traces focuses on finding information leakage in HTTP traces and identifying connection attempts to command-and-control servers. Conversions containing International Mobile Equipment Identity number (IMEI), phone number or credit card information were tracked. If no abnormalities are detected so far, the packet dump is compared manually to a dump generated by the uninfected VM template image to determine whether the sample was not detected or simply inactive.
\item \emph{Strength of the work:}\\ 
1. Blacklisting DNS servers for possible malicious repository.\\
2. String matching on HTTP header flags, GET, POST requests for possible malicious data transfer. \\
3. The researchers were able to observe client-side communication using mock DNS and HTTP server responses. In total, 18 samples were investigated, generating traces compared against the patterns of identifying information. Of those, 8 samples were detected, 2 evaded detection and 8 samples failed to execute in the virtualized environment. 
\item \emph{Limitations:}\\ 
1. Addresses cannot be black-listed until a malicious application is identified and
the connections it makes are analyzed. \\
2. Use of Android x86 virtual machines also introduces several limitations. An Android x86 VM is not a cellphone, it does not support text messages, and has a non-standard IMEI and IMSI. Requests of IMEI and IMSI return the null value.
\end{itemize}

\subsection{Detection of Malicious Android Mobile Applications Based on Aggregated System Call Events \cite{mrr:dmamabasce}} 
\begin{itemize}
\item The research suggests a method to distinguish Android-based malicious apps based on the system call event pattern internally activated after running suspicious malicious applications. It analyzed the malicious system call event pattern selected from Android Malware Genome Project. The actual system call patterns are extracted from the normal and malicious apps on Android-based mobile devices. And then, feature events were aggregated to calculate a similarity analysis between normal and malicious event set.
\item \emph{Strength of the work:} \\
1. Found pattern of normal and malicious system call events.\\
2. classifies apps based on malicious behavior based on system call events.\\
3. makes activity pattern comparison between normal and malicious apps.\\
4. Established Quantifiable similarity among malware examples.
\item \emph{Limitations:}\\
Identifies 17 system call events which do not occur in normal application, are
found in malicious applications. Therefore, any given apps could be suspected as malicious mobile application if the 17 kinds of system call events above have occurred simultaneously in the application. But in our research we found out that there is at most 2/3 system calls that can correspond to such decision. 
\end{itemize}



\subsection{Identifying android malicious repackaged applications by thread-grained system call sequences \cite{mrr:iamratscs}} 
\begin{itemize}
\item Based on Malicious Repackaged Applications (MRAs), this work proposes a mechanism SCSdroid (System Call Sequence Droid), which adopts the thread-grained system call sequences activated by applications. The concept is that even if MRAs can be camouflaged as benign applications, their malicious behavior would still appear in the system call sequences. SCSdroid extracts the truly malicious common subsequences from the system call sequences of MRAs belonging to the same family. Therefore, these extracted common subsequences can be used to identify any evaluated application without requiring the original benign application. 
\item \emph{Strength of the work:} SCSdroid achieved up to 95.97\% detection accuracy, i.e., 143 correct detections among 149 applications.
\item \emph{Limitations:} No significant limitations.
\end{itemize}

\section{Difference between our approach and existing ones}

The aforementioned researches focuses on various malware detection attempts. Here follows a comparison of those approaches outlining difference between ours and those. 

\begin{itemize}
\item The First two research \cite{mrr:mmedd} and \cite{mrr:dmmw}, were targeted on defining malwares and classification of their detection methods, where our work avoids such attempts and was targeted on refining detection methods and implementing them.
\item The research work by Y. Zhou et al.\cite{mrr:damce} was focused on detecting malwares by identifying specific system calls (They devised 17 such),  whereas our work proposes system call frequencies to detect malicious behavior in an application.
\item The work that follows after\cite{mrr:damnl} represents first definitive approach among we have analyzed with its focus on network level analysis. Our work uses a similar but more specific approach like packet capture and filtering destination for black-listed domains.
\item The last two works put their concentration on aggregated system call events \cite{mrr:dmamabasce} and system call patterns in repackaged applications \cite{mrr:iamratscs}, respectively. In contrast, our work tries to form a concrete and generalized approach by establishing a threshold for acceptance based on ``system call frequency''.      
\end{itemize}

In short, there are notable differences among the presented works before and in the next chapter, our first approach, the network based analysis is illustrated outlining our work to its full proportion.  
